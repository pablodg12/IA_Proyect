\documentclass[spanish, fleqn]{article}
\usepackage[utf8]{inputenc}
\usepackage{graphicx}
\usepackage{amsmath}
\usepackage{bm}
\usepackage{tikz}
\usepackage{adjustbox}
\usepackage[top=3cm,bottom=3cm,left=3.5cm,right=3.5cm,footskip=1.5cm,headheight=1.5cm,headsep=.5cm,textheight=3cm]{geometry}

\begin{document}
\title{Inteligencia Artificial \\ \begin{Large}Estado del Arte: The Progressive Party Problem\end{Large}}
\author{Pablo Ibarra S.}
\date{\today}
\maketitle

\section*{Evaluación}

\begin{tabular}{ll}
Resumen (5\%): & \underline{\hspace{2cm}} \\
Introducción (5\%):  & \underline{\hspace{2cm}} \\
Definición del Problema (10\%):  & \underline{\hspace{2cm}} \\
Estado del Arte (35\%):  & \underline{\hspace{2cm}} \\
Modelo Matemático (20\%): &  \underline{\hspace{2cm}}\\
Conclusiones (20\%): &  \underline{\hspace{2cm}}\\
Bibliografía (5\%): & \underline{\hspace{2cm}}\\
 &  \\
\textbf{Nota Final (100\%)}:   & \underline{\hspace{2cm}}
\end{tabular}

\begin{abstract}
En este informe se presenta un estado del arte del problema conocido como Progressive Party Problem[citar 1]. El problema consiste en programar de la mejor manera una fiesta de yates, donde distintas personas visitaran a un o varios yates anfitriones. Ademas se definen terminos y conceptos importantes utilizados en la literatura e investigaciones en el área. Se explica y define el problema a tratar, así como también se realizan descripciones y clasificaciones de las distintas variantes que se han desarrollado a través de los años. Se describen las distintas estrategias y algoritmos que se han desarrollado para la resolución del problema principal junto a sus variantes y se presentan también modelos matemticos del problema.
\end{abstract}

\section{Introducción}

Hoy en dia la programacion de horarios es un tema vital para personas y organizaciones, debido a que si se realiza bien, los recursos que se disponen y utilizan se distribuiran de la  manera mas eficiente e inteligente. La programación lineal entera a sido una herramienta clave durante mucho tiempo para resolver este tipo de problemas, aun asi, existen ciertos problemas que PLE(glo) no puede resolver debido a su exploción combinatorial(glo), un ejemplo problema que presenta estas caracteristicas es el problema llamado Progressive Party Problem y sera nuestro foco de estudio en el presente informe.

El problema conocido como Progressive Party Problem fue introducido por Peter Hubbard miembro de la asociación de propietarios SeaWych(link) y del departamento de matematicas de la universidad de Southampton(link), cuando tuvo que organizar una fiesta de yates en la isla Wight, que se encuentra al sur de la costa de Inglaterra. El problema nos situa en el contexto de una fiesta de yates durante la tarde, donde hay $\mathit{n}$ yates con sus tripulaciones. Una cantidad $\mathrm{A}$ de botes seran anfitriones: en ellos se realizaran fiestas donde la tripulacion de los demas yates, que en este contexto los nombraremos como tripulacion huesped, los visitaran en intervalos de tiempo $\mathit{t}$ = 1...$\mathit{T}$ de media hora cada uno, la cantidad de periodos $\mathit{T}$ esta previamente definida. Las tripulaciones huespedes iran visitando cada uno de los botes anfitriones, respetando la capacidad que estos tienen junto a las reglas de que no pueden visitar un bote anfitrion mas de una vez y encontrarse con la misma tripulación huesped mas de una vez durante la fiesta.

El proposito de este informe es investigar sobre dichos problemas, sobre los metodos que existen para solucionarlo, presentar antedeces de lo que se a desarrollado, presentar hacia donde apuntan las nuevas investigaciones
 
\section{Bibliografía}
Indicando toda la informaci\'on necesaria de acuerdo al tipo de documento revisado. Todas las referencias deben ser citadas en el documento.
\bibliographystyle{plain}
\bibliography{Referencias}


\end{document} 