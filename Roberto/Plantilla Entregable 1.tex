\documentclass[letter, 10pt]{article}
\usepackage[utf8]{inputenc}
\usepackage[spanish]{babel}
\usepackage{amsfonts}
\usepackage{amsmath}
\usepackage[dvips]{graphicx}
\usepackage{url}
\usepackage[top=3cm,bottom=3cm,left=3.5cm,right=3.5cm,footskip=1.5cm,headheight=1.5cm,headsep=.5cm,textheight=3cm]{geometry}


\begin{document}
\title{Inteligencia Artificial \\ \begin{Large}Estado del Arte: Progressive Party Problem\end{Large}}
\author{[Roberto Fuentes Zenteno]}
\date{\today}
\maketitle


%--------------------No borrar esta secci\'on--------------------------------%
\section*{Evaluaci\'on}

\begin{tabular}{ll}
Resumen (5\%): & \underline{\hspace{2cm}} \\
Introducci\'on (5\%):  & \underline{\hspace{2cm}} \\
Definici\'on del Problema (10\%):  & \underline{\hspace{2cm}} \\
Estado del Arte (35\%):  & \underline{\hspace{2cm}} \\
Modelo Matem\'atico (20\%): &  \underline{\hspace{2cm}}\\
Conclusiones (20\%): &  \underline{\hspace{2cm}}\\
Bibliograf\'ia (5\%): & \underline{\hspace{2cm}}\\
 &  \\
\textbf{Nota Final (100\%)}:   & \underline{\hspace{2cm}}
\end{tabular}
%---------------------------------------------------------------------------%
\vspace{2cm}


\begin{abstract}
Se presenta el problema \textit{Progressive Party Problem} (PPP), propuesto por Peter Hubbard (\textit{Southampton
University}). Consiste en organizar de la mejor manera una ``fiesta progresiva'' en un grupo de yates, en donde durante periodos consecutivos de tiempo la tripulación de los yates invitados debe visitar los yates anfitriones. El objetivo del problema es asignar todas las tripulaciones visitantes a los yates anfitriones para cada período de tiempo, minimizando la cantidad de yates anfitriones. Se describe el problema, luego se analiza el estado actual del problema mediante el estado del arte, seguido de un modelo matemático, además de las técnicas que se han usado para abordarlo y los resultados acotados que se han obtenido. 
\textbf{Keywords}: PPP, optimización combinatoria, programación lineal entera, programación de restricciones.
\end{abstract}

\section{Introducci\'on}
Una conferencia es una manera excelente en que las personas que tienen intereses comunes pueden reunirse e intercambiar las ideas m\'as vanguardistas de su campo. Es por esto que la gestión de eventos progresivos es un elemento importante a controlar y mejorar. En este documento, se presenta el \textit{Progressive Party Problem} (PPP), problema especifico que plante\'o Peter Hubbard de \textit{Southampton University} en el a\~no 1996~\cite{PPP_ILP_CPC}, cuyo objetivo es asignar todas las tripulaciones visitantes a los yates anfitriones para cada período de tiempo, minimizando la cantidad de yates anfitriones. El documento proporciona el planteamiento del problema bien definido junto con su formulación a través de Programación entera y Programación de restricciones junto a una explicación de como fueron los resultados de ambos experimentos. Además, existen otros documentos donde se entrega la data del problema y sus respectiva solución.
\\

\textit{Progressive Party Problem} (PPP) busca encontrar la mejor manera de gestionar yates invitados y anfitriones, donde los yates invitados podrán invitar a los anfitriones en determinados periodos de tiempo. Su objetivo es minimizar los yates anfitriones respetando ciertas restricciones, como evitar exceder la cantidad de los yates anfitriones o que las tripulaciones invitadas no podrán toparse más de una vez.
\\

La motivación del problema surge debido al problema originado en Bembridge, Isla de Wight~\cite{PPP_MIP}. Su contexto es organizar un programa social para un paseo de yates, donde estos estaban amarrados en un puerto deportivo. Para que las personas se reúnan con tantos otros asistentes como le fuera posible, se planeó una fiesta nocturna en donde algunos de estos yates serian designados como anfitriones, y las tripulaciones de los barcos restantes ir\'ian visitando a estos yates anfitriones durante seis horas en periodos de media hora. El problema que enfrentaba el organizador del paseo era el de minimizar el número de los barcos anfitriones, ya que cada anfitrión tenía que ser abastecido con alimentos y otros prerrequisitos. 
\\

La estructura del actual trabajo es la siguiente: En la sección siguiente se define y se explica detalladamente el problema \textit{Progressive Party Problem}, en la sección 3 se plantea el estado del arte del problema analizado, indicando los soluciones encontradas hasta el d´ıa de hoy. Luego en la sección 4 se presenta el modelo matemático que mejor resuelve el problema con el fin de que cualquier persona sea capaz de interpretarlo. Finalmente en la sección 5 se concluye sobre el problema estudiado.

\section{Definición del Problema}
\subsection{Progressive Party Problem}
\subsubsection{Definición}
\textit{Progressive Party Problem} (PPP) busca encontrar la mejor manera de gestionar yates invitados y anfitriones, donde los yates invitados podrán invitar a los anfitriones en determinados periodos de tiempo. 

\subsubsection{Objetivo}
El objetivo de PPP es asignar todas las tripulaciones visitantes a los yates anfitriones para cada período de tiempo, minimizando la cantidad de yates anfitriones.

\subsubsection{Parametros}
\textit{Progressive Party Problem} tiene los siguientes parámetros:

\begin{itemize}
\item $H$: Cantidad de yates.
\item $K_i$: Capacidad del yate $i$.
\item $T$: cantidad de períodos.
\item $c_i$: cantidad de tripulantes del yate $i$.
\item $H_i$: 1 si el yate $i$ es un anfitrión.
\item $G_{ikt}$ = 1 si el yate $k$ es un invitado del yate $i$ en el período $t$
\item $m_{klt}$ = 1 si el yate $k$ y $l$ se encuentran en el período $t$.
\end{itemize}

\subsubsection{Variables}
Algunas variables de PPP son las siguentes:

    \begin{itemize}
      \item $\delta_i$: 1 si el yate $i$ es un yate anfitrión.
      \item $\gamma_{ikt}$: 1 si el yate $k$ es un invitado del bote $i$ en el periodo $t$.
    \end{itemize}

\subsubsection{Restricciones}
\begin{itemize}
\item Un yate solo podrá ser visitado si es un yate anfitrión.
\item La capacidad de los yates anfitriones no se puede exceder.
\item Cada tripulación debe siempre tener un anfitrión asociado o ser uno.
\item Una tripulación invitada no podrá visitar un yate anfitrión más de una vez.
\item Las tripulaciones invitadas no podrán toparse más de una vez.
\end{itemize}

\subsection{Otras variantes}
Tenemos varios problemas relacionados con la gestión de yates, los cuales buscan distintas maneras de resolver el problema: Como un CSP, minimizar la cantidad de yates anfitriones, maximizar la duración de la fiesta, definir diferentes restricciones del problema. Las variantes clasicas son:

\begin{enumerate}
    \item CSP PPP: Ver si es factible o no realizar una fiesta con las características dadas. ~\cite{PPP_ILP_CPC_CPP}
    
    \item rellenar aqui 
\end{enumerate}

\section{Estado del Arte}




\section{Modelo Matem\'atico}
Uno o m\'as modelos matem\'aticos para el problema, idealmente indicando el espacio de b\'usqueda para cada uno. Cada modelo debe estar correctamente referenciado, adem\'as no debe ser una imagen extraida. Tambi\'en deben explicarse en detalle cada una de las partes, mostrando claramente la funci\'on a maximizar/minimizar, variables y restricciones. Tanto las f\'ormulas como las explicaciones deben ser consistentes.

\section{Conclusiones}
Conclusiones RELEVANTES del estudio realizado. Deber\'ia responder a las preguntas: ?`todas las t\'ecnicas resuelven el mismo problema o hay algunas diferencias?, ?`En qu\'e se parecen o difieren las t\'ecnicas en el contexto del problema?, ?`qu\'e limitaciones tienen?, ?`qu\'e t\'ecnicas o estrategias son las m\'as prometedoras?, ?`existe trabajo futuro por realizar?, ?`qu\'e ideas usted propone como lineamientos para continuar con investigaciones futuras?


\section{Bibliograf\'ia}
Indicando toda la informaci\'on necesaria de acuerdo al tipo de documento revisado. Todas las referencias deben ser citadas en el documento.
\bibliographystyle{plain}
\bibliography{Referencias}

\end{document} 