\documentclass[spanish, fleqn]{article}
\usepackage[utf8]{inputenc}
\usepackage{graphicx}
\usepackage{amsmath}
\usepackage{bm}
\usepackage{tikz}
\usepackage{adjustbox}
\usepackage[top=3cm,bottom=3cm,left=3.5cm,right=3.5cm,footskip=1.5cm,headheight=1.5cm,headsep=.5cm,textheight=3cm]{geometry}

\begin{document}
\title{Inteligencia Artificial \\ \begin{Large}Estado del Arte: The Progressive Party Problem\end{Large}}
\author{Pablo Ibarra S.}
\date{\today}
\maketitle

\section*{Evaluación}

\begin{tabular}{ll}
Resumen (5\%): & \underline{\hspace{2cm}} \\
Introducción (5\%):  & \underline{\hspace{2cm}} \\
Definición del Problema (10\%):  & \underline{\hspace{2cm}} \\
Estado del Arte (35\%):  & \underline{\hspace{2cm}} \\
Modelo Matemático (20\%): &  \underline{\hspace{2cm}}\\
Conclusiones (20\%): &  \underline{\hspace{2cm}}\\
Bibliografía (5\%): & \underline{\hspace{2cm}}\\
 &  \\
\textbf{Nota Final (100\%)}:   & \underline{\hspace{2cm}}
\end{tabular}

\begin{abstract}
En este informe se presenta un estado del arte del problema conocido como Progressive Party Problem[citar 1]. El problema consiste en programar de la mejor manera una fiesta de yates, donde distintas personas visitaran a un o varios yates anfitriones. Ademas se definen terminos y conceptos importantes utilizados en la literatura e investigaciones en el área. Se explica y define el problema a tratar, así como también se realizan descripciones y clasificaciones de las distintas variantes que se han desarrollado a través de los años. Se describen las distintas estrategias y algoritmos que se han desarrollado para la resolución del problema principal junto a sus variantes y se presentan también modelos matemticos del problema.
\end{abstract}

\section{Introducción}

Hoy en día 

\section{Bibliograf\'ia}
Indicando toda la informaci\'on necesaria de acuerdo al tipo de documento revisado. Todas las referencias deben ser citadas en el documento.
\bibliographystyle{plain}
\bibliography{Referencias}


\end{document} 